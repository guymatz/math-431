\documentclass[10pt]{article}
\usepackage[utf8]{inputenc}
\usepackage[T1]{fontenc}
\usepackage{amsmath}
\usepackage{amsfonts}
\usepackage{amssymb}
\usepackage[version=4]{mhchem}
\usepackage{stmaryrd}

\title{UW - Math 431 \\
Probability Theory \\
Extra Homework 4}

\author{Guy Matz}
\date{\today}


\begin{document}


\hfill \break
Exercise 2.14. Let $A$ and $B$ be two disjoint events. Under what condition are they independent?

\hfill \break
Exercise 2.16. We flip a fair coin three times. For $i=1,2,3$, let $A_{i}$ be the event that among the first $i$ coin flips we have an odd number of heads. Check whether the events $A_{1}, A_{2}, A_{3}$ are independent or not.

\hfill \break
Exercise 2.20. A fair die is rolled repeatedly. Use precise notation of probabilities of events and random variables for the solutions to the questions below.

(a) Write down a precise sum expression for the probability that the first five rolls give a three at most two times.

(b) Calculate the probability that the first three does not appear before the fifth roll.

(c) Calculate the probability that the first three appears before the twentieth roll but not before the fifth roll.

\hfill \break
Exercise 2.30. Assume that $\frac{1}{3}$ of all twins are identical twins. You learn that Miranda is expecting twins, but you have no other information.

(a) Find the probability that Miranda will have two girls.

(b) You learn that Miranda gave birth to two girls. What is the probability that the girls are identical twins?

Explain any assumptions you make.


\hfill \break
Exercise 2.52. Suppose that $P(A)=0.3, P(B)=0.2$, and $P(C)=0.1$. Further, $P(A \cup B)=0.44, P\left(A^{c} C\right)=0.07, P(B C)=0.02$, and $P(A \cup B \cup C)=0.496$. Decide whether $A, B$, and $C$ are mutually independent.


\hfill \break
Exercise 2.58. Suppose that a person's birthday is a uniformly random choice from the 365 days of a year (leap years are ignored), and one person's birthday is independent of the birthdays of other people. Alex, Betty and Conlin are comparing birthdays. Define these three events:

$$
\begin{aligned}
& A=\{\text { Alex and Betty have the same birthday }\} \\
& B=\{\text { Betty and Conlin have the same birthday }\} \\
& C=\{\text { Conlin and Alex have the same birthday }\} .
\end{aligned}
$$

(a) Are events $A, B$ and $C$ pairwise independent? (See the definition of pairwise independence on page 54 and Example 2.25 for illustration.)

(b) Are events $A, B$ and $C$ independent?


\hfill \break
Exercise 3.4. Let $X \sim \operatorname{Unif}[4,10]$.

(a) Calculate $P(X<6)$.

(b) Calculate $P(|X-7|>1)$.

(c) For $4 \leq t \leq 6$, calculate $P(X<t \mid X<6)$.


\hfill \break
Exercise 3.26. Suppose that $X$ is a discrete random variable with possible values $\{1,2, \ldots\}$, and probability mass function

$$
p_{X}(k)=\frac{c}{k(k+1)}
$$

with some constant $c>0$.

(a) What is the value of $c$ ?

Hint. $\frac{1}{k(k+1)}=\frac{1}{k}-\frac{1}{k+1}$.

(b) Find $E(X)$.

Hint. Example D.5 could be helpful.

\end{document}

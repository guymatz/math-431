\documentclass[10pt]{article}
\usepackage[utf8]{inputenc}
\usepackage[T1]{fontenc}
\usepackage{amsmath}
\usepackage{amsfonts}
\usepackage{amssymb}
\usepackage[version=4]{mhchem}
\usepackage{stmaryrd}
\usepackage{tikz}
%\usepackage{enumitem}
\usepackage[inline]{enumitem}

%\input{.tex/preamble}
%\input{.tex/macros}
%\input{.tex/letterfonts}

\title{UW - Math 431 \\
Probability Theory \\
Homework 3}

\author{Guy Matz}
\date{\today}


\begin{document}
\maketitle

\begin{itemize}

    \item[3.7]  Suppose that the continuous random variable $X$ has cumulative distribution function given by

$$
F(x)= \begin{cases}0, & \text { if } \quad x<\sqrt{2} \\ x^{2}-2, & \text { if } \quad \sqrt{2} \leq x<\sqrt{3} \\ 1, & \text { if } \quad \sqrt{3} \leq x .\end{cases}
$$
    \begin{enumerate}
      \item  Find the smallest interval $[a, b]$ such that of $P(a \leq X \leq b)=1$.

        \[ ( \sqrt{2}, \sqrt{3} ) \]
      \item  Find $P(X=1.6)$.

          \[ 0 \]
      \item  Find $P\left(1 \leq X \leq \frac{3}{2}\right)$.
        \begin{align*}
          F(\frac{3}{2}) - (1) &= F(\frac{3}{2}) - (1) \\
                               &= F(\frac{3}{2}) - (\sqrt{2}) \\
                               &= \frac{9}{4} - 2 - (2 -2) = \frac{1}{4} 
        \end{align*}

      \item  Find the probability density function of $X$.
          \[ f(x) = F'(x) = 2x \]
    \end{enumerate}

\newpage
    \item[3.12]  Suppose that $X$ is a random variable taking values in $\{1,2,3, \ldots\}$ with probability mass function
      \[ p_{X}(n)=\frac{6}{\pi^{2}} \cdot \frac{1}{n^{2}} \]

Show that $E[X]=\infty$.

Hint. See Example D.5.
      \begin{align*}
        \sum^{\infty}_{n=1} n \cdot P(X=n) &= \sum^{\infty}_{n=1} \\
            &= \sum^{\infty}_{n=1} n \frac{6}{\pi^2} \cdot \frac{1}{n^2} \\
            &= \sum^{\infty}_{n=1} \frac{6}{n \pi^2} \\
            &= \frac{6}{\pi^2} \sum^{\infty}_{n=1} \frac{1}{n} \\
            &= \frac{6}{\pi^2} \cdot \infty \\
            &= \infty
      \end{align*}


\newpage
    \item[3.15]  Suppose that the random variable $X$ has expected value $E[X]=$ 3 and variance $\operatorname{Var}(X)=4$. Compute the following quantities.

    \begin{enumerate}
      \item  $E[3 X+2]$
            \begin{align*}
              E[3X +2] &= 3E[X] + 2 \\
                       &= 3 \cdot 3 + 2 \\
                       &= 11
            \end{align*}
      \item  $E\left[X^{2}\right]$
            \begin{align*}
              E[X^2] &=  Var(x) + (E[X])^2 \\
                    &=  4 + 3^2 \\
                    &= 13
            \end{align*}

      \item  $E[(2 X+3)^{2}]$
            \begin{align*}
              E[(2X+3)^2] &= E[4X^2 + 12X + 9] \\
                         &= 4E[X^2] + 12 E[X] + 9 \\
                         &= 4 \cdot 13 + 12 \cdot 3 + 9 \\
                         &= 52 + 36 + 9 \\
                         &= 97
            \end{align*}

      \item  $\operatorname{Var}(4 X-2)$
          \[ Var(4X-2) = 16 \cdot Var(X) = 16 \cdot 4 = 64 \]
    \end{enumerate}

\newpage
    \item[3.17]  Let $X$ be a normal random variable with mean $\mu=-2$ and variance $\sigma^{2}=7$. Find the following probabilities using the table in Appendix E.

    \begin{enumerate}
      \item  $P\left(X>3.5\right)$
        \begin{align*}
          P\left(X > 3.5\right) &= P\left(\frac{X+2}{\sqrt{7}}  > \frac{3.5+2}{\sqrt{7}}\right)\\
            &=  \Phi\left(-2.078\right) \\
            &=  1- \Phi\left(2.078\right) \\
            &=  1- 0.9812 \\
            &=  0.0188
        \end{align*}
      \item  $P\left(-2.1<X<-1.9\right)$
        \begin{align*}
          P\left(-2.1<  X < -1.9\right) &= P\left(\frac{X+2}{\sqrt{7}}  < \frac{-1.9+2}{\sqrt{7}}\right) - P\left(\frac{X+2}{\sqrt{7}}  < \frac{-2.1+2}{\sqrt{7}}\right)\\
            &=  P\left(\frac{X+2}{\sqrt{7}}  < \frac{.1}{\sqrt{7}}\right) - P\left(\frac{X+2}{\sqrt{7}}  < \frac{-.1}{\sqrt{7}}\right)\\
            &=  \Phi\left(0.0378\right) - \left(1 - \Phi\left(0.0378\right) \right) \\
            &=  0.032
        \end{align*}

      \item  $P\left(X<2\right)$
        \begin{align*}
          P\left(X < 2\right) &= P\left(\frac{X+2}{\sqrt{7}}  < \frac{2+2}{\sqrt{7}}\right)\\
            &=  \Phi\left(1.5118\right) \\
            &=  0.9357
        \end{align*}

      \item  $P\left(X<-10\right)$
        \begin{align*}
          P\left(X < 2\right) &= P\left(\frac{X+2}{\sqrt{7}}  < \frac{-10+2}{\sqrt{7}}\right)\\
            &=  \Phi\left(-3.023\right) \\
            &=  1- \Phi\left(3.023\right) \\
            &=  1 - .9987 \\
            &= 0.0013
        \end{align*}

      \item  $P\left(X>4\right)$
        \begin{align*}
          P\left(X > 4\right) &= P\left(\frac{X+2}{\sqrt{7}}  > \frac{4+2}{\sqrt{7}}\right)\\
            &=  1- \Phi\left(2.2267\right) \\
            &=  1 - .9881 \\
            &= 0.12
        \end{align*}
    \end{enumerate}


\newpage
    \item[3.68]  Let $Z \sim \mathcal{N}(0,1)$ and $X \sim \mathcal{N}\left(\mu, \sigma^{2}\right)$.

    \begin{enumerate}
      \item  Calculate $E\left[Z^{4}\right]$.

        Sorry for the abbreviated answer here.  It's a lot to type!
        \begin{align*}
          E[Z^4] &= \frac{1}{\sqrt{2\pi}} \int_{-\infty}^{\infty} x^4 \frac{1}{\sqrt{2\pi}}
            e^{-\frac{x^2}{2}}  d{x} \\
                 &= \frac{1}{\sqrt{2\pi}} -x^3 e^{-\frac{x^2}{2}} \biggr\rvert_{-\infty}^{\infty} + 3 \sqrt{2\pi} \\
                &= 3
        \end{align*}

      \item  Calculate $E\left[X^{4}\right]$.

        Using Moment-Generating Function
        \begin{align*}
          M_X^4(t) &= E[e^{4tX}] \\
                   &= E \left[ exp \left(\frac{4t}{\sigma \sqrt{n}}
                      \sum^{n}_{i=1} (X_i - \mu \right) \right] \\
                   &= E \left[ \prod_{i=1}^n exp \left(\frac{4t}{\sigma \sqrt{n}}
                      (X_i - \mu \right) \right] \\
                  &= ???
        \end{align*}
    \end{enumerate}


\newpage
    \item[3.70]  Let $X \sim \mathcal{N}\left(\mu, \sigma^{2}\right)$ and $Y=a X+b$. By adapting the calculation from (3.42) and (3.43), show that $Y \sim \mathcal{N}\left(a \mu+b, a^{2} \sigma^{2}\right)$.

        \begin{align}
          F(Y) &= P(Y=y) \\
            &= P(a \sigma + b \mu < y) \\
            &= P(X < \frac{y-b \mu}{a \sigma} \\
            &= \Phi( \frac{y-b \mu}{a \sigma} )
        \end{align}
        And
        \begin{align}
          f(y) &= F'(y) \\
               &= \frac{d}{dx} \left[ \Phi \left( \frac{y-b \mu}{a \sigma} \right) \right] \\
               &= \frac{1}{a\sigma} \phi \left( \frac{y-b\mu}{a\sigma} \right) \\
               &= \frac{1}{\sqrt{2 \pi a^2 \sigma^2}} e^{- \frac{(y-b\mu)^2}{2a^2 \sigma^2}} \\
              &= ???
        \end{align}

\newpage
    \item[3.73]  Let $Y$ have density function

    \[ f_{Y}(x)=
      \begin{cases}
        \frac{1}{2} x^{-2}, & x \leq-1 \text { or } x \geq 1 \\
       0, & -1<x<1
     \end{cases}
      \]
    Is it possible to give $E[Y]$ a meaningful value?

    \begin{align*}
      E[Y] &= \int_{-\infty}^{-1} x \cdot \frac{1}{2x^2} d{x} + \int_{1}^{\infty} x \cdot \frac{1}{2x^2} d{x} \\
          &= -\infty + \infty \\
          &= 0
    \end{align*}

    So $E[Y]=0$.  Even though our function is not defined at $x=0$, 
    this is still a meaningful value.  FFor according to the book,
    "An expectation $E(X)$ is well defined if it has a definite value,
    either a finite number or positive or negative infinity.


\newpage
    \item[4.20]  You flip a fair coin 10,000 times. Approximate the probability that the difference between the number of heads and number of tails is at most 100.

        \begin{align*}
          P\left(4950 < X < 5050\right) &= 2 \cdot P\left(\frac{X+2}{\sqrt{7}}  > \frac{3.5+2}{\sqrt{7}}\right)\\
        \end{align*}

\newpage
    \item[9.2]  Let $X$ be an exponential random variable with parameter $\lambda=\frac{1}{2}$.

      \[ E[X] = 2, Var(X) = 4 \]
    \begin{enumerate}
      \item  Use Markov's inequality to find an upper bound for $P(X>6)$.

        \[ P(X \geq 6) \leq \frac{E[X]}{6} = \frac{2}{6} = \frac{1}{3}  \]

      \item  Use Chebyshev's inequality to find an upper bound for $P(X>6)$.
        \[ P(X \geq 6) \leq \frac{Var(X)}{6^2} = \frac{4}{36} = \frac{1}{9}  \]

      \item  Explicitly compute the probability above and compare with the upper bounds you derived.

        \begin{align*}
          P(X > 6) &= 1 - (1 - -e^{-\frac{6}{2}}) \\
                   &= e ^{-3}\\
                   &\approx 0.05
        \end{align*}
        Mine is better than the Russki guys's!
    \end{enumerate}


\newpage
    \item[9.20]  Let $X_{1}, X_{2}, X_{3}, \ldots$ be i.i.d. random variables with mean zero and finite variance $\sigma^{2}$. Let $S_{n}=X_{1}+\cdots+X_{n}$. Determine the limits below, with precise justifications.

    \begin{enumerate}
      \item  $\lim_{n \rightarrow \infty} P\left(S_{n} \geq 0.01 n\right)$

        Due to the Law of Large Numbers, the limit will approach 0

      \item  $\lim_{n \rightarrow \infty} P\left(S_{n} \geq 0\right)$

        The sum will be bounce around 0, so the probability that it
        will be positive is .5

      \item  $\lim_{n \rightarrow \infty} P\left(S_{n} \geq-0.01 n\right)$

        Due to the Law of Large Numbers, the limit will approach 1
    \end{enumerate}
\end{itemize}
\end{document}

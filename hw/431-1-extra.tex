\documentclass[10pt]{article}
\usepackage[utf8]{inputenc}
\usepackage[T1]{fontenc}
\usepackage{amsmath}
\usepackage{amsfonts}
\usepackage{amssymb}
\usepackage[version=4]{mhchem}
\usepackage{stmaryrd}

\title{UW - Math 431 \\
Probability Theory \\
Homework 1}

\author{Guy Matz}
\date{\today}


\begin{document}
\maketitle

\begin{itemize}

\item 1.2: For breakfast Bob has three options: cereal, eggs or fruit. He has to choose exactly two items out of the three available.

\begin{enumerate}
  \item  Describe the sample space of this experiment.

Hint. What are the different possible outcomes for Bob's breakfast?

  \item  Let $A$ be the event that Bob's breakfast includes cereal. Express $A$ as a subset of the sample space.

\end{enumerate}

\hfill \break
\item 1.6: We have an urn with 3 green and 4 yellow balls. We choose 2 balls randomly without replacement. Let $A$ be the event that we have two different colored balls in our sample.

(a) Describe a possible sample space with equally likely outcomes, and the event $A$ in your sample space.

(b) Compute $P(A)$.

\hfill \break
\item 1.14: Assume that $P(A)=0.4$ and $P(B)=0.7$. Making no further assumptions on $A$ and $B$, show that $P(A B)$ satisfies $0.1 \leq P(A B) \leq 0.4$.

\hfill \break
\item 1.18: The statement \textit{SOME DOGS ARE BROWN}

has 16 letters. Choose one of the 16 letters uniformly at random. Let $X$ denote the length of the word containing the chosen letter. Determine the possible values and probability mass function of $X$.

\hfill \break
\item 1.22: We pick a card uniformly at random from a standard deck of 52 cards. (If you are unfamiliar with the deck of 52 cards, see the description above Example C.19 in Appendix C.)

(a) Describe the sample space $\Omega$ and the probability measure $P$ that model this experiment.

(b) Give an example of an event in this probability space with probability $\frac{3}{52}$.

(c) Show that there is no event in this probability space with probability $\frac{1}{5}$.

\hfill \break
\item 1.26: 10 men and 5 women are meeting in a conference room. Four people are chosen at random from the 15 to form a committee.

(a) What is the probability that the committee consists of 2 men and 2 women?

(b) Among the 15 is a couple, Bob and Jane. What is the probability that Bob and Jane both end up on the committee?

(c) What is the probability that Bob ends up on the committee but Jane does not?

\hfill \break
\item 1.32: You are dealt five cards from a standard deck of 52. Find the probability of being dealt a full house. (This means that you have three cards of one face value and two cards of a different face value. An example would be three queens and two fours. See Exercise C.10 in Appendix C for a description of all the poker hands.)

\hfill \break
\item 1.34: Pick a uniformly chosen random point inside a unit square (a square of sidelength 1) and draw a circle of radius $1 / 3$ around the point. Find the probability that the circle lies entirely inside the square.

\hfill \break
\item 1.40: An urn contains 1 green ball, 1 red ball, 1 yellow ball and 1 white ball. I draw 4 balls with replacement. What is the probability that there is at least one color that is repeated exactly twice?

Hint. Use inclusion-exclusion with events $G=$ \{exactly two balls are green\}, $R=\{$ exactly two balls are red $\}$, etc.

\hfill \break
\item 1.46: An urn contains 3 red balls and 1 yellow ball. We draw balls from the urn one by one without replacement until we see the yellow ball. Let $X$ denote the number of red balls we see before the yellow. Find the possible values and the probability mass function of $X$.

\hfill \break
\item 1.48: Consider the experiment of drawing a point uniformly at random from the unit interval $[0,1]$. Let $Y$ be the first digit after the decimal point of the chosen number. Explain why $Y$ is discrete and find its probability mass function.

\end{itemize}
\end{document}

\documentclass[10pt]{article}
\usepackage[utf8]{inputenc}
\usepackage[T1]{fontenc}
\usepackage{amsmath}
\usepackage{amsfonts}
\usepackage{amssymb}
\usepackage[version=4]{mhchem}
\usepackage{stmaryrd}
\usepackage{tikz}
%\usepackage{enumitem}
\usepackage[inline]{enumitem}

%\input{.tex/preamble}
%\input{.tex/macros}
%\input{.tex/letterfonts}

\title{UW - Math 431 \\
Probability Theory \\
Homework 5}

\author{Guy Matz}
\date{\today}


\begin{document}
\maketitle

\begin{itemize}

	\item[5.24] Let $X \sim \mathcal{N}(0,1)$ and $Y=e^{X} . Y$ is called a log-normal random variable.

	\begin{enumerate}
		\item  Find the probability density function of $Y$.

			For $y \leq 0$ we have $P(Y \leq t) = 0$.  For $y \geq 0$,
			\begin{align*}
				F_Y(y) &= P(Y \leq y)\\
					&= P(X \leq \ln(y))\\
					&= \Phi(\ln(y))
			\end{align*}
		\item  Find the $n$th moment $E\left(Y^{n}\right)$ of $Y$.

Hint. Do not compute the moment generating function of $Y$. Instead relate the $n$th moment of $Y$ to an expectation of $X$ that you know.

	\[ M_Y(t) = e^{te^X} \]
	So the nth moment is
		\[ M^n_Y(t) = \left(t e^X \right)^n e^{te^X} \]
	\end{enumerate}
\newpage
	\item[5.29] Let $Z \sim \mathcal{N}(0,1)$. Find the probability density function of $|Z|$.

		For $y < 0, P(Y \leq y) =0$.  For $y \geq 0$
		\begin{align*}
			F_Y(y) &= P(Y \leq y) \\
					&= P(|Z| \leq y)\\
					&= \Phi(y) - \Phi(-y)\\
					&= \Phi(y) - (1 - \Phi(y))\\
					&= 2\Phi(y) - 1 \\
		\end{align*}
		So
		\begin{align*}
			f_Y(y) &= F_Y'(y) \\
					&= 2 \Phi'(y) \\
					&= 2 \phi(y) \cdot \frac{1}{2y}  \\
					&= \frac{\phi(y)}{y} 
		\end{align*}

\newpage
	\item[5.31] Suppose that $U \sim \operatorname{Unif}[0,1]$. Let $Y=e^{\frac{U}{1-U}}$. Find the probability density function of $Y$.

		$Y$ is increasing on $[0,1]$ and the function
		$g(y) = \frac{U}{e^{1-U}}$ is one-to-one, so we can use Fact
		5.24 which states:
		\[ f_Y(y) = f_X(g^{-1}(y) \cdot \frac{1}{|g'(g^{-1}(y))|}  \]

		The only problem is that I can't figure out how to get $g^{-1}$!
\newpage
	\item[6.2] The joint probability mass function of the random variables $(X, Y)$ is given by the following table:

\begin{center}
\includegraphics[max width=\textwidth]{images/exercise-6.2.png}

\end{center}

	\begin{enumerate}
		\item  Find the marginal probability mass functions of $X$ and $Y$.
			\begin{align*}
				P(X=1)=\frac{10}{30} \\
				P(X=2)=\frac{15}{30} \\
				P(X=3)=\frac{5}{30} \\
			\end{align*}
			\begin{align*}
				P(Y=0)=\frac{6}{30} \\
				P(Y=1)=\frac{6}{30} \\
				P(Y=2)=\frac{10}{30} \\
				P(Y=3)=\frac{8}{30} \\
			\end{align*}
		\item  Calculate the probability $P\left(X+Y^{2} \leq 2\right)$.
			\[ P(1,0) + P(2,0) + P(1,1) = \frac{2}{30} + \frac{3}{30} +  \frac{2}{30} = \frac{7}{30}  \]
	\end{enumerate}

\newpage
	\item[6.3] For each lecture the professor chooses between white, yellow, and purple chalk, independently of previous choices. Each day she chooses white chalk with probability 0.5 , yellow chalk with probability 0.4 , and purple chalk with probability 0.1 .

	\begin{enumerate}
		\item  What is the probability that over the next 10 days she will choose white chalk 5 times, yellow chalk 4 times, and purple chalk 1 time?
		\[ X \sim \operatorname{Mult}(10,3,.5,.4,.1)  \]
		\begin{align*}
			P(X_W=5, X_Y=4, X_P = 1) &= \binom{10}{5,4,1} \cdot \left(.5\right)^5 \cdot \left(.4\right)^4 \cdot \left(.1\right)^1 \\
			&= \frac{10!}{5!4!1!} \cdot \left(.5\right)^5 \cdot \left(.4\right)^4 \cdot \left(.1\right)^1 \\
			&= 10 \cdot 3 \cdot 7 \cdot 6 \cdot \left(.5\right)^5 \cdot \left(.4\right)^4 \cdot \left(.1\right)^1 \\
			&= 0.1
		\end{align*}

		\item  What is the probability that over the next 10 days she will choose white chalk exactly 9 times?
			\begin{align*}
				P(X_W = 9) &= \binom{10}{9,1,0} \cdot \left( .5 \right)^9 \cdot \left( .4 \right)^1 \cdot \left(.1\right)^0 \\
				             &+ \binom{10}{9,0,1} \cdot \left(.5\right)^9 \cdot \left(.4\right)^0 \cdot \left(.1\right)^1 \\
							 &= 0.0097
			\end{align*}
	\end{enumerate}


\newpage
	\item[6.6] Suppose that $X, Y$ are jointly continuous with joint probability density function


\[
f(x, y)= \begin{cases}x e^{-x(1+y)}, & \text { if } x>0 \text { and } y>0 \\ 0, & \text { otherwise. }\end{cases}
\]
		\emph{Hint. For the last two parts, recalling moments of the exponential distribution can be helpful.}

	\begin{enumerate}
		\item  Find the marginal density functions of $X$ and $Y$.
			\begin{align*}
				f_X(x) &= \int_{{0}}^{{\infty}} {xe^{-x(1+y)}} d{y} \\
					&= -e^{-x(1+y)}|_0^{\infty} \\
					&= -(0 - e^{-x} )\\
					&= e^{-x}
			\end{align*}
			\begin{align*}
				f_Y(y) &= \int_{{0}}^{{\infty}} {xe^{-x(1+y)}} d{x} \\
					   &= -\frac{xe^{-x(1+y)}}{1+y} + \frac{e^{-x(1+y)}}{(1+y)^2} |_0^{\infty} \\
					&= -(0 - 0 - (0 + \frac{1}{(1+y)^2}  \\
					&= -\left(0 - 0 - \left(0 + \frac{1}{(1+y)^2}\right) \right) \\
					&= \frac{1}{(1+y)^2}  \\
			\end{align*}

		\item  Calculate the expectation $E[X Y]$.
			\begin{align*}
				E[XY] &= \int_{0}^{\infty} \left( \int_{0}^{\infty} x^2ye^{-x(1+y)} d{y} \right) d{x} \\
					  &= \frac{-ye^{-x(1+y)}}{x} - \frac{e^{-x(1+y)}}{x^2} |_0^{\infty} \\
					  &= 1
			\end{align*}

		\item  Calculate the expectation $E\left[\frac{X}{1+Y}\right]$.
			\begin{align*}
				E\left[\frac{X}{1+Y}\right] &= \int_{0}^{\infty} \left( \int_{0}^{\infty} \frac{x^2ye^{-x(1+y)}}{1+y} d{x} \right) d{y} \\
				&= \int_0^{\infty} \left(
				\frac{-x^2e^{-x(1+y)}}{1+y}
			+ \frac{2xe^{-x(1+y)}}{(1+y)^2}
			+ \frac{2e^{-x(1+y)}}{(1+y)^3}
			|_0^{\infty}
		\right) dy\\
			&= \int_0^{\infty}
			\left( 
			\frac{2}{(1+y)^3}
			\right) dy\\
			&= 
			- \frac{1}{ (1+y)^2}
			|_0^{\infty} \\
			&= 1
			\end{align*}

			
	\end{enumerate}

\newpage
	\item[6.18] Suppose that $X$ and $Y$ are integer-valued random variables with joint probability mass function given by

\[
p_{X, Y}(a, b)= \begin{cases}\frac{1}{4 a}, & \text { for } \quad 1 \leq b \leq a \leq 4 \\ 0, & \text { otherwise. }\end{cases}
\]
	\begin{enumerate}
		\item  Show that this is indeed a joint probability mass function.

			\begin{align*}
				p_X(x) &= \sum^{}_{y}p_{X,Y}(x,y)  \\
					&= p(1,1) + p(1,2) \dots p(3,4) + p(4,4)\\
					&= \frac{1}{4} + 2 \frac{1}{8} + 3\frac{1}{12} + 4 \frac{1}{16} \\
					&= 1
			\end{align*}
			\begin{align*}
				p_Y(y) &= \sum^{}_{x}p_{X,Y}(x,y)  \\
					&= p(1,1) + p(1,2) \dots p(3,4) + p(4,4)\\
					&= \frac{1}{4} + 2 \frac{1}{8} + 3\frac{1}{12} + 4 \frac{1}{16} \\
					&= 1
			\end{align*}
		\item  Find the marginal probability mass functions of $X$ and $Y$.
			\begin{align*}
				P(X=1) &= \frac{1}{4} 
				P(X=2) &= \frac{1}{4} 
				P(X=3) &= \frac{1}{4} 
				P(X=4) &= \frac{1}{4} 
			\end{align*}
			\begin{align*}
				P(Y=1) &= \frac{1}{2} 
				P(Y=2) &= \frac{1}{4} 
				P(Y=3) &= \frac{1}{8} 
				P(Y=4) &= \frac{1}{16} 
			\end{align*}

		\item  Find $P(X=Y+1)$.
				\begin{align*}
					P(X=Y+1) &= \frac{1}{8} +\frac{1}{12} +\frac{1}{16}\\
							&= \frac{13}{48} 
				\end{align*}
	\end{enumerate}

\newpage
	\item[6.22] Let $\left(X_{1}, X_{2}, X_{3}, X_{4}\right) \sim \operatorname{Mult}\left(n, 4, p_{1}, p_{2}, p_{3}, p_{4}\right)$. What is the distribution of $X_{1}+X_{2}$ ?

Hint. This can be done without any computation. Take a look at the argument at the end of Example 6.10 .

\[ (X_1 + X_2) \sim  \operatorname{Mult}\left(2n, 4, p_{1}, p_{2}, p_{3}, p_{4}\right) \]

\newpage
	\item[6.37] Let $h$ be a continuous function on $[a, b]$ such that $h(a)=h(b)=0$ and $h(x)>0$ for $a<x<b$. Let $D$ be the region between the $x$-axis and the graph of $h$ :
	\[ D=\{(x, y): a<x<b, 0<y<h(x)\} \]
Let $(X, Y)$ be a uniformly distributed random point on $D$. Find the marginal density function $f_{X}$ of $X$.

I don't see how to do this without knowing $h(0)$ \dots

\newpage
	\item[6.42] Let $(X, Y)$ be a uniform random point on the rectangle

$$
D=[0,2] \times[0,3]=\{(x, y): 0 \leq x \leq 2,0 \leq y \leq 3\} .
$$

Let $Z=X+Y$. Give the joint cumulative distribution function of $(X, Z)$.

\emph{Hint. Recall the definition of the cumulative distribution function and draw a picture.}
\[
P(X \leq x, X + Y \leq z) =
\begin{cases}
0, & \text { if } z < 0 \\
\frac{1}{6} \cdot \frac{z^2}{2}, & \text { for } 0 \leq z \leq x \\
\frac{1}{6} \cdot \left( \frac{z^2}{2} -3z \frac{x^3}{2}  + 3x^2 \right) , & \text { for } x \leq z \leq 3 \\
\frac{1}{6} \cdot \left( 3z - z^2 - 9 \right) & \text { for } 3 \leq z \leq 3 + x \\
1, & \text { if } z > 5
\end{cases}
\]

\end{itemize}
\end{document}

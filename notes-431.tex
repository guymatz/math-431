\documentclass{report}

\input{.tex/preamble}
\input{.tex/macros}
\input{.tex/letterfonts}

\title{
  \Huge{Math 431 - Introduction to Probability Theory}
  \\
  Notes
}
\author{\huge{Guy Matz}}
\date{}

\begin{document}
%\maketitle

% \setcounter{chapter}{1}
\chapter*{20230619 - Introduction}

\begin{itemize}
  \item Reiew of set nontation and DeMorgan's rule (Appendix B, C)

    \begin{itemize}

    \item $\Omega$ is a set and contains some elements.  if $\Omega$
      contains $\omega$ we write $\omega \in \Omega$

    \item A is a subset of B: $A \subset B$

    Union, interection, complement . . .\\
    Assume A and B are subsets of $\Omega$
    $$A \cup B = \{ \omega \in \Omega: \omega \in A OR \omega \in B\}$$
    $$A \cap B = \{ \omega \in \Omega: \omega \in A AND \omega \in B\}$$
    $$A^c = \{ \omega \in \Omega: \omega \notin A \}$$
    $$A \setminus B=\{\omega \in \Omega:\omega \in A AND \omega \notin B\}$$
    \[ A \setminus B = A \cap B^c \]

    DeMorgan's Law: $\left( \cap A_i \right)^c = \cup^{\infty}_{i=1} A_i^c$

    \end{itemize}

  \item Definitions
    \dfn{ Probability }{
      Mathematical models of experiments (See video @ min 32)
    }

    \dfn{ Sample space  }{
      $\Omega$: set of possible outcomes
    }

    \dfn{ Sample Points }{
      $\omega$: Elements of $\Omega$ are \underline{Sample points}
    }

    \dfn{ Event Space }{
      $\mathcal{F}$: Subsets of $\Omega$
    }

    \dfn{ Probability Measure }{
    $P$: Probability Measure == Probability Distribution == Probability
    }

    \dfn{ Probability Space }{
      $(\Omega, \mathcal{F}, P)$
    }
  \end{itemize}

\chapter*{20230620 - Random Sampling}
\begin{itemize}
  \item Order - objects chosen one ata time
  \item unOrder - objects chosen in a group
  \item With Replacement - 
  \item WithOUT Replacement - 
  \item ORDERed with replacement
    \[ P(\omega) = \left( \frac{1}{\#\Omega} \right) = \left( \frac{1}{n} \right)^k \]
  \item ORDERed withOUT replacement (falling factorial)
  \[ P(\omega) = \frac{1}{(n)_k} = \frac{k!}{n!} \]
  \[ \#\Omega = (n)_k \]
  \[ \#A = \text{\# of slots for each  option} \cdot \text{\# of each option} \]
  \item UNordered withOUT replacement - "chosen all at once"
    \[ \#\Omega = \binom{n}{k} = \frac{n!}{(n-k)!k!}  \]
    \[ \#A = \prod_{j=0}^{m} \binom{n_j}{k_j} \]

\end{itemize}

\chapter*{20230622}%
\section*{Random Varaibles}%

  \dfn{ Random Variable }{
    A function from $\Omega$ into the real numbers
  }
  \dfn{ Probability Distribution }{
    The Collection of probabilites $P\{X \in B \}$ for sets $B \subset R$ 

    An assignment of probabilities to subsets of $R$ that satisfies
    the axioms of probability
  }
  \dfn{ Discrete RV }{
    an RV is discrete if there exists a finite or countably infinite
    set $\{k_1, k_2,,,\}$ such that the sum ranges 

    bodyRead up on this
  }
  \dfn{ Probability Mass Function - p.m.f }{
    \[ p(k) = P(X=k) \]
    Describes a discrete RV.  The function $p_X$ gives the probability
    of each possible value of X.
  }

  \dfn{ Conditional Prob }{
    Let BB be an event in the sample space such that $P(B) > 0$.
    THen for all events A, the conditional prob of "A given B" is
    deffined by 
    \[ P(A|B) = \frac{P(A \cap B)}{P(B)} = \frac{P(A B)}{P(B)}  \]
  }

  \dfn{ Partition }{
    A finite collection off events $\{B_1..B_n\}$ is a \underline{partition}
    off $\Omega $ if the sets $B_i$ are pairwise disjoint and together
    they make up $\Omega$.  That is, $B_iB_j = \emptyset$ when $i \neq
    j$ and $\cup^n_{i=1}B_i = \Omega$
  }

  



\chapter*{20230627}%
  \begin{itemize}
    \item Independence
      \begin{enumerate}
        \item get 3 different version
      \end{enumerate}
  \end{itemize}
\end{document}

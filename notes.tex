\documentclass{report}

\input{.tex/preamble}
\input{.tex/macros}
\input{.tex/letterfonts}

\title{
  \Huge{Math 431 - Introduction to Probability Theory}
  \\
  Notes
}
\author{\huge{Guy Matz}}
\date{}

\begin{document}
%\maketitle

\setcounter{chapter}{1}
\chapter{Conditional Probability and Independence}

\setcounter{section}{3}
\section{Independent Trials}

\subsection{Discrete Distributions}
\begin{itemize}
  \item \textbf{Bernoulli}
     \begin{itemize}
       \item $X \sim \text{Ber}(p)$.
       \item Records the result of a single trial with two possible outcomes
     \end{itemize}
  \item \textbf{Binomial}
    \begin{itemize}
    \item $X \sim \text{Bin}(n,p)$
    \item Counts successes
    \item $P(X=k) = \binom{n}{k} p^k (1-p)^{n-k}$ for $k = 0,1, \ldots , n$
    \item Examples
      \begin{itemize}
        \item What is the probability that five rolls of a fair die yield two or three
sixes?\\\\
          The repeated trial is a roll of the die, and success means rolling a six. Let $S_5$ be
          the number of sixes that appeared in five rolls. Then $S_5 \sim Bin(5, 1/6)$.\\\\
          \begin{aligned}
            P\left(S_5 \in\{2,3\}\right) & =P\left(S_5=2\right)+P\left(S_5=3\right) \\
              & = \binom{5}{2} \left(\frac{1}{6}\right)^2 \left(\frac{5}{6}\right)^3 +
                  \binom{5}{3} \left(\frac{1}{6}\right)^3 \left(\frac{5}{6}\right)^2
          \end{aligned}
      \end{itemize}
    \end{itemize}
  \item \textbf{Geometric}
    \begin{itemize}
    \item $X \sim \text{Geom}(p)$
    \item Probability of $k$ failures before a success
    \item $P(N = k) = (1-p)^{k-1}p$
    \item Examples
      \begin{itemize}
        \item What is the probability that it takes more than seven rolls of a fair
die to roll a six?\\\\
          Let N be the number of rolls of a fair die until the first six. Then $N \sim$ Geom(1/6).
      \[
          P(N>7) = \sum_{k=8}^{\infty} P(N=k)
          = \sum_{k=8}^{\infty}(\frac{5}{6})^{k-1} \frac{1}{6}
          =\frac{1}{6} (\frac{5}{6}) ^7 \sum_{j=0}^{\infty}(\frac{5}{6})^j
          =\frac{\frac{1}{6} (\frac{5}{6} )^7}{1-\frac{5}{6}}
          =(\frac{5}{6})^7
        \]
    \item Roll a pair of fair dice until you get either a sum of 5 or a sum of 7. What is the probability that you get 5 first?\\\\
      Let $A$ be the event that 5 comes first, and let
      $A_n=\{$ no 5 or 7 in rolls $1, \ldots, n-1$, and 5 at roll $n\}$
      The events $A_n$ are pairwise disjoint and their union is $A$. Some simple calculations:\\
      $P($ pair of dice gives 5$)=\frac{4}{36} \quad$ and $\quad P($ pair of dice gives 7$)=\frac{6}{36}$.
      By the independence of the die rolls
      $$
      P\left(A_n\right)=\left(1-\frac{10}{36}\right)^{n-1} \frac{4}{36}=\left(\frac{13}{18}\right)^{n-1} \frac{1}{9}
      $$
      Consequently
      \[
        P(A)=\sum_{n=1}^{\infty} P\left(A_n\right)
            =\sum_{n=1}^{\infty}\left(\frac{13}{18}\right)^{n-1} \frac{1}{9}
            =\frac{\frac{1}{9}}{1-\frac{13}{18}}
            =\frac{2}{5}
      \]
      \end{itemize}
    \end{itemize}
  \item \textbf{HyperGeometric}
    \begin{itemize}
    \item $X \sim \text{Hypergeom}(N, N_A, n)$
    \item Probability, $P(X = k)$, of sampling $k$ items of type $A$ ($N_A$)
      without replacement from a total of $N$.  $N = N_A + N_B$, $n$ is the total number 
      in the sample.
    \item $P(X = k) = \frac{ \binom{N_A}{k} \binom{N - N_A}{n -k} }{ \binom{N}{n} }$ for $k = 0,1, \dots,n$
    \item Examples
      \begin{itemize}
        \item A basket contains a litter of 6 kittens, 2 males and 4 females.
          A neighbor comes and picks 3 kittens randomly to take home with him. Let
          X be the number of male kittens in the group the neighbor chose.
          Then $X \sim$ Hypergeom(6,2,3), with p.m.f.\\
         $P(X = 0) = \frac{ \binom{2}{0} \binom{6 - 2}{3 - 0} }{ \binom{6}{3} } = \frac{1}{5}
         \qquad
          P(X = 1) = \frac{ \binom{2}{1} \binom{6 - 2}{3 - 1} }{ \binom{6}{3} } = \frac{3}{5}$\\
         $P(X = 2) = \frac{ \binom{2}{2} \binom{6 - 2}{3 - 2} }{ \binom{6}{3} } = \frac{1}{5}
         \qquad
          P(X = 3) = \frac{ \binom{3}{3} \binom{6 - 3}{3 - 3} }{ \binom{6}{3} } = 0$
      \end{itemize}
    \end{itemize}
  \end{itemize}



\section{Further Topics on Sampling and Independence}
  \begin{itemize}
    \item Conditional Independence: If events $A1,A2, \cdots , A_n$ are independent under the measure $P( \cdot | B)$
      \[ P(A_i_1 A_i_2 \cdots A_i_k | B) = P(A_i_1 | B) P(A_i_2 | B) \cdots P(A_i_k | B) \]
    \item Examples
      \begin{itemize}
        \item Suppose after testing positive, a person is retested for the disease
and this test also comes back positive. What is the probability of disease after two
positive tests?
        \[ \begin{aligned}
            P\left(D \mid A_1 A_2\right)
            & =\frac{P\left(D A_1 A_2\right)}{P\left(A_1 A_2\right)}
              =\frac{P\left(A_1 A_2 \mid D\right) P(D)}{P\left(A_1 A_2 \mid D\right)
                P(D)+P\left(A_1 A_2 \mid D^c\right) P\left(D^c\right)} \\
            & =\frac{P\left(A_1 \mid D\right) P\left(A_2 \mid D\right)
                P(D)}{P\left(A_1 \mid D\right) P\left(A_2 \mid D\right)
                P(D)+P\left(A_1 \mid D^c\right) P\left(A_2 \mid D^c\right) P\left(D^c\right)}
           \end{aligned} \]
        \item 
      \end{itemize}
  \end{itemize}


\chapter{Random Variables}
  \section{Probability Distributions}
    \begin{itemize}
      \item Probability Mass Function
        \begin{itemize}
          \item A function from the set of possible discrete values of X into [0, 1].
          \item $ \sum^{}_{k} p_X(k) = \sum^{}_{k}  P(X=k) = 1 $
          \item $ P(X \in B) =  \sum^{}_{k \in B} p_X(k) $
        \end{itemize}
      \item Probability Density Function
        \begin{itemize}
          \item A function from an interval of possible real values of X into [0, 1].
          \item $ P(X \leq b) =  \int^{b}_{-\infty} f(x) dx $
          \item Distributions
              \begin{itemize}
                \item Uniform
                    \begin{itemize}
                      \item $X \sim$ Unif$[a,b]$
                      \item $f(x) = \frac{1}{b-a} $ for $x \in [a,b]$
                      \item If $X \sim$ Unif$[a,b]$ and $[c,d] \subset [a,b]$, then
                        $ P(c \leq X \leq d) = \int_{{c}}^{{d}} {\frac{1}{b-a}} \: d{x} {} $
                    \end{itemize}
              \end{itemize}
        \end{itemize}
    \end{itemize}

\end{document}

\documentclass[10pt]{article}
\usepackage[utf8]{inputenc}
\usepackage[T1]{fontenc}
\usepackage{amsmath}
\usepackage{amsfonts}
\usepackage{amssymb}
\usepackage[version=4]{mhchem}
\usepackage{stmaryrd}
\usepackage{tikz}
%\usepackage{enumitem}
\usepackage[inline]{enumitem}

%\input{.tex/preamble}
%\input{.tex/macros}
%\input{.tex/letterfonts}

\title{UW - Math 431 \\
Probability Theory \\
Homework 1}

\author{Guy Matz}
\date{\today}


\begin{document}

\begin{enumerate}
  \item How many different ways can we choose 3 different positive
    integers between 1 and 11 if we do not care about the order
    of the numbers? (This means that the choice 1,2,3 is considered
    to be the same as the choice 3,1,2.)

    \[ \binom{11}{3} = 165 \]

  \item How many different ways can we choose 3 different positive
    integers between 1 and 9 if we do not care about the order
    of the numbers? (This means that the choice 1,2,3 is considered
    to be the same as the choice 3,1,2.)

    \[ \binom{9}{3} = 84 \]

\newpage
  \item We roll a die 4 times and then flip a coin 3 times. How many
    different outcomes can we have for this experiment where there
    are no two die rolls with the same number and not all coin flips
    are tails?

    \[ 6 \cdot 5 \cdot 4 \cdot 3 \cdot (2^3 - 1) = 2520 \]

  \item We roll a die 2 times and then flip a coin 2 times. How many
    different outcomes can we have for this experiment where there
    are no two die rolls with the same number and not all coin flips
    are tails?

    \[ 6 \cdot 5 \cdot (2^2 - 1) = 90 \]

\newpage
  \item Let $q=\sum_{k=1}^{\infty}\frac{1}{2^{ak+b}}$,
    where $a=4$ and $b=3$.  Find the exact value of $\frac{1}{q}$.

    120 ?

\newpage
  \item Suppose that $A, B, \text{ and } C$ are all subsets of the set 
    $\Omega$.  Which of the following expressions describe the set of
    elements in $\Omega$ that are in $C$, but neither in $A$
    nor in $B$?
    \begin{align}
      \checkmark & (A \cup B)^c \cap C \\
      & A^c \cup B^c \cup C \\
      \checkmark & A^c \cap B^c \cap C \\
      \checkmark & C \backslash(A \cup B) \\
      & \left(A^c \cup B^c\right) \cap C \\
      & (A \cap B)^c \cap C
    \end{align}

\newpage
  \item We roll two fair dice and denote the sum of the two numbers by
    $X$. What is the size of the event $\{X = x\}$ if $x=4$, and we
    use the sample space 
    \[ \{(a_1, a_2): 1 \leq a_1, a_2 \leq 6\} \]

    \[ A = \{ \{ 1,3 \}, \{2,2\}, \{3,1\} \} \]
      \[ \#A = 3 \]

\newpage
  \item Choose the statements that are true for any 
    $(\Omega, \mathcal{F}, \mathbb{P})$ probability space.

    \begin{enumerate}
      \item If $A$ is an event and $P(A)=1 / 3$ then $P\left(A^c\right)=2 / 3$$\checkmark$
      \item If $A$ and $B$ are events then $P(A \cup B)$ is always at least as large than $P(A)$.$\checkmark$
      \item If $A$ is an event and $P(A)=0$ then $A=\emptyset.\chi$
      \item If $A$ and $B$ are events and $A \cup B=\Omega$ then $P(A)+P(B)=1\chi$
    \end{enumerate}
\end{enumerate}
\end{document}
